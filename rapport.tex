\documentclass[12pt,a4paper]{article}
\usepackage[utf8]{inputenc}
\usepackage[T1]{fontenc}
\usepackage[french]{babel}
\usepackage{geometry}
\usepackage{graphicx}
\usepackage{xcolor}
\usepackage{titlesec}
\usepackage{fancyhdr}
\usepackage{lastpage}
\usepackage{amsmath}
\usepackage{amssymb}
\usepackage{array}
\usepackage{booktabs}
\usepackage{longtable}
\usepackage{tabularx}
\usepackage{multirow}
\usepackage{makecell}
\usepackage{enumitem}
\usepackage{listings}
\usepackage{minted}
\usepackage{hyperref}
\usepackage{caption}
\usepackage{subcaption}
\usepackage{float}
\usepackage{tikz}
\usetikzlibrary{shapes,arrows,positioning,calc}
\usepackage{pgfplots}
\usepackage{siunitx}
\usepackage{rotating}
\usepackage{lscape}
\usepackage{afterpage}

% Couleurs professionnelles
\definecolor{primary}{RGB}{0,51,102}
\definecolor{secondary}{RGB}{0,102,153}
\definecolor{accent}{RGB}{255,102,0}
\definecolor{lightgray}{RGB}{240,240,240}
\definecolor{darkgray}{RGB}{100,100,100}

% Configuration de la page
\geometry{
    top=2.5cm,
    bottom=2.5cm,
    left=2.5cm,
    right=2.5cm,
    headheight=1.5cm,
    footskip=1.5cm
}

% En-têtes et pieds de page
\pagestyle{fancy}
\fancyhf{}
\renewcommand{\headrulewidth}{0.5pt}
\renewcommand{\footrulewidth}{0.5pt}
\fancyhead[L]{\color{primary}\normalfont\small\textbf{CareerPath AI}}
\fancyhead[C]{\color{primary}\normalfont\small\textbf{Rapport Technique}}
\fancyhead[R]{\color{darkgray}\normalfont\small\thepage/\pageref{LastPage}}
\fancyfoot[L]{\color{darkgray}\normalfont\footnotesize Réalisé par Hamdi Abdallah}
\fancyfoot[C]{\color{darkgray}\normalfont\footnotesize \today}
\fancyfoot[R]{\color{darkgray}\normalfont\footnotesize Version 1.0}

% Formatage des sections
\titleformat{\section}
{\color{primary}\normalfont\Large\bfseries}
{\thesection}{1em}{}

\titleformat{\subsection}
{\color{secondary}\normalfont\large\bfseries}
{\thesubsection}{1em}{}

\titleformat{\subsubsection}
{\color{darkgray}\normalfont\normalsize\bfseries}
{\thesubsubsection}{1em}{}

% Configuration des listes
\setlist[itemize]{leftmargin=*, label=\color{accent}\textbullet}
\setlist[enumerate]{leftmargin=*, label=\color{accent}\arabic*.}

% Configuration des tableaux
\newcolumntype{Y}{>{\raggedright\arraybackslash}X}
\newcolumntype{Z}{>{\centering\arraybackslash}X}

% Configuration du code
\lstset{
    basicstyle=\ttfamily\footnotesize,
    backgroundcolor=\color{lightgray},
    frame=single,
    framesep=3pt,
    rulecolor=\color{gray!30},
    breaklines=true,
    postbreak=\mbox{\textcolor{accent}{$\hookrightarrow$}\space},
    showstringspaces=false,
    tabsize=2,
    captionpos=b
}

% Pour les diagrammes
\tikzset{
    block/.style = {draw, fill=white, rectangle, minimum height=2em, minimum width=4em},
    input/.style = {coordinate},
    output/.style = {coordinate},
    pinstyle/.style = {pin edge={to-,thin,black}},
    decision/.style = {draw, fill=white, diamond, aspect=2},
    arrow/.style = {draw, -latex'},
    cloud/.style = {draw, ellipse,fill=lightgray, minimum height=2em},
    startstop/.style = {rectangle, rounded corners, draw, fill=red!30, minimum width=3cm, minimum height=1cm, text centered},
    process/.style = {rectangle, draw, fill=blue!30, minimum width=3cm, minimum height=1cm, text centered},
    io/.style = {trapezium, trapezium left angle=70, trapezium right angle=110, draw, fill=green!30, minimum width=3cm, minimum height=1cm},
    database/.style = {cylinder, aspect=0.5, shape border rotate=90, draw, fill=orange!30, minimum width=1.5cm, minimum height=2cm}
}

\begin{document}

% Page de titre
\begin{titlepage}
    \centering
    \vspace*{2cm}
    
    {\Huge\color{primary}\textbf{RAPPORT TECHNIQUE}}\\[1.5cm]
    {\LARGE\color{secondary}\textbf{CareerPath AI}}\\[0.5cm]
    {\Large\color{darkgray}\textbf{Plateforme de Recrutement Intelligente}}\\[1.5cm]
    
    \rule{\linewidth}{0.5mm}\\[0.5cm]
    {\huge\bfseries\color{primary}Architecture et Spécifications Techniques}\\[0.5cm]
    \rule{\linewidth}{0.5mm}\\[2cm]
    
    {\Large\color{darkgray}\textbf{Réalisé par}}\\[0.5cm]
    {\LARGE\color{secondary}\textbf{Hamdi Abdallah}}\\[2cm]
    
    \vfill
    
    \begin{minipage}{0.4\textwidth}
        \begin{flushleft}
            \large
            \textbf{Date:}\\
            \today\\
        \end{flushleft}
    \end{minipage}
    \hfill
    \begin{minipage}{0.4\textwidth}
        \begin{flushright}
            \large
            \textbf{Version:}\\
            1.0
        \end{flushright}
    \end{minipage}
    
    \vspace{2cm}
    

\end{titlepage}

% Table des matières
\tableofcontents
\newpage

% Liste des figures et tableaux
\listoffigures
\listoftables
\newpage

% Résumé exécutif
\section{Résumé Exécutif}
\label{sec:resume}

\begin{abstract}
\textbf{CareerPath AI} est une plateforme innovante de recrutement qui intègre l'intelligence artificielle locale pour améliorer le processus de matching candidat-offre. Ce document présente l'architecture complète, les spécifications techniques, les modèles de données et la stratégie de mise en œuvre de la solution. La plateforme combine une stack technologique moderne (MERN) avec une IA locale via Ollama/Llama 3.2, offrant ainsi confidentialité des données et réduction des coûts opérationnels.
\end{abstract}

\subsection*{Points Clés}
\begin{itemize}
    \item Architecture full-stack MERN avec séparation MVC claire
    \item Intégration IA locale via Ollama avec fallback
    \item Authentification JWT avec vérification OTP par email
    \item Gestion des rôles (candidat, recruteur, administrateur)
    \item Génération automatique de lettres de motivation
    \item Analyse de compatibilité candidat-offre
    \item Tests property-based pour garantir la correction
    \item API RESTful complètement documentée
\end{itemize}

\subsection*{Bénéfices Attendus}
\begin{table}[H]
    \centering
    \begin{tabularx}{\textwidth}{|l|X|}
        \hline
        \rowcolor{lightgray}
        \textbf{Aspect} & \textbf{Bénéfice} \\
        \hline
        Confidentialité & Traitement local des données sensibles \\
        \hline
        Coûts & Élimination des frais d'API cloud externes \\
        \hline
        Performance & Latence réduite pour les opérations IA \\
        \hline
        Évolutivité & Architecture modulaire et extensible \\
        \hline
        Fiabilité & Tests approfondis garantissant la stabilité \\
        \hline
    \end{tabularx}
    \caption{Bénéfices principaux de la solution}
    \label{tab:benefices}
\end{table}

\section{Introduction}
\label{sec:introduction}

\subsection{Contexte du Projet}
Le marché du recrutement évolue vers des solutions plus intelligentes et automatisées. CareerPath AI répond à ce besoin en proposant une plateforme qui combine les avantages des systèmes traditionnels avec la puissance de l'IA locale, garantissant ainsi confidentialité et contrôle total sur les données.

\subsection{Objectifs du Projet}
\begin{enumerate}
    \item Fournir une plateforme de recrutement complète avec gestion des utilisateurs, entreprises, offres et candidatures
    \item Intégrer l'IA locale pour des fonctionnalités avancées sans dépendance externe
    \item Assurer une expérience utilisateur fluide et intuitive
    \item Garantir la sécurité et la confidentialité des données
    \item Maintenir des performances optimales à grande échelle
\end{enumerate}

\subsection{Portée du Document}
Ce document couvre:
\begin{itemize}
    \item L'architecture technique complète
    \item Les spécifications fonctionnelles détaillées
    \item Les modèles de données
    \item La stratégie de test
    \item Le plan d'implémentation
\end{itemize}

\section{Architecture du Système}
\label{sec:architecture}

\subsection{Vue d'Ensemble}
L'architecture suit le pattern MVC (Modèle-Vue-Contrôleur) avec une séparation claire entre les différentes couches de l'application.

\begin{figure}[H]
    \centering
    \begin{tikzpicture}[node distance=2cm, auto]
        % Frontend
        \node[block, fill=blue!20, minimum width=3cm, minimum height=2cm] (frontend) {Frontend \\ React.js};
        
        % Backend
        \node[block, fill=green!20, minimum width=3cm, minimum height=2cm, right=3cm of frontend] (backend) {Backend \\ Node.js/Express};
        
        % Database
        \node[database, fill=orange!20, below right=1cm and 2cm of backend] (db) {MongoDB};
        
        % AI Service
        \node[cloud, fill=purple!20, above right=1cm and 2cm of backend] (ai) {Ollama \\ Llama 3.2};
        
        % Arrows
        \draw[arrow, thick] (frontend) -- node[above] {API REST} (backend);
        \draw[arrow, thick] (backend) -- node[right] {Mongoose} (db);
        \draw[arrow, thick] (backend) -- node[above] {HTTP} (ai);
        
        % Layers
        \node[above=0.5cm of frontend] {\textbf{Couche Présentation}};
        \node[above=0.5cm of backend] {\textbf{Couche Métier}};
        \node[below=0.5cm of db] {\textbf{Couche Données}};
        \node[above=0.5cm of ai] {\textbf{Service IA}};
    \end{tikzpicture}
    \caption{Architecture globale du système}
    \label{fig:architecture}
\end{figure}

\subsection{Architecture Backend Détaillée}
\begin{figure}[H]
    \centering
    \begin{tikzpicture}[node distance=1.5cm, auto]
        % User Request
        \node[io, fill=blue!30] (request) {Requête HTTP};
        
        % Route Layer
        \node[process, fill=green!30, below of=request] (routes) {Routes};
        
        % Middleware Layer
        \node[process, fill=yellow!30, below of=routes] (middleware) {Middleware \\ (Auth, Validation)};
        
        % Controller Layer
        \node[process, fill=orange!30, below of=middleware] (controller) {Controllers};
        
        % Service Layer
        \node[process, fill=red!30, below of=controller] (service) {Services Métier};
        
        % Repository Layer
        \node[process, fill=purple!30, below of=service] (repository) {Repository};
        
        % Database
        \node[database, fill=gray!30, below of=repository] (database) {MongoDB};
        
        % Arrows
        \draw[arrow] (request) -- (routes);
        \draw[arrow] (routes) -- (middleware);
        \draw[arrow] (middleware) -- (controller);
        \draw[arrow] (controller) -- (service);
        \draw[arrow] (service) -- (repository);
        \draw[arrow] (repository) -- (database);
        
        % Response arrows
        \draw[arrow, dashed] (database) -- node[right] {Données} (repository);
        \draw[arrow, dashed] (repository) -- (service);
        \draw[arrow, dashed] (service) -- (controller);
        \draw[arrow, dashed] (controller) -- (middleware);
        \draw[arrow, dashed] (middleware) -- (routes);
        \draw[arrow, dashed] (routes) -- node[left] {Réponse} (request);
    \end{tikzpicture}
    \caption{Flux de données backend}
    \label{fig:backend-flow}
\end{figure}

\subsection{Structure des Répertoires}
\begin{lstlisting}[caption=Structure des répertoires backend, label=lst:structure]
backend/
├── src/
│   ├── config/           # Configuration
│   │   ├── database.js
│   │   ├── email.js
│   │   ├── ollama.js
│   │   └── constants.js
│   ├── middleware/       # Middleware
│   │   ├── auth.js
│   │   ├── roleGuard.js
│   │   └── upload.js
│   ├── models/          # Modèles Mongoose
│   │   ├── User.js
│   │   ├── CandidateProfile.js
│   │   ├── Company.js
│   │   ├── JobOffer.js
│   │   ├── Application.js
│   │   └── Skill.js
│   ├── controllers/     # Contrôleurs
│   │   ├── authController.js
│   │   ├── profileController.js
│   │   ├── companyController.js
│   │   ├── jobController.js
│   │   ├── applicationController.js
│   │   ├── skillController.js
│   │   └── aiController.js
│   ├── services/        # Services métier
│   │   ├── authService.js
│   │   ├── emailService.js
│   │   ├── profileService.js
│   │   ├── companyService.js
│   │   ├── jobService.js
│   │   ├── applicationService.js
│   │   ├── skillService.js
│   │   └── aiService.js
│   ├── routes/          # Routes API
│   │   ├── auth.js
│   │   ├── profile.js
│   │   ├── company.js
│   │   ├── jobs.js
│   │   ├── applications.js
│   │   ├── skills.js
│   │   └── ai.js
│   ├── utils/           # Utilitaires
│   │   ├── pdfParser.js
│   │   ├── otpUtils.js
│   │   ├── validators.js
│   │   └── errors.js
│   └── app.js          # Point d'entrée
├── tests/              # Tests
│   ├── unit/
│   ├── integration/
│   └── properties/
├── uploads/           # Fichiers uploadés
├── .env.example       # Variables d'environnement
├── package.json       # Dépendances
└── README.md          # Documentation
\end{lstlisting}

\section{Spécifications Techniques}
\label{sec:specifications}

\subsection{Stack Technologique}
\begin{table}[H]
    \centering
    \begin{tabularx}{\textwidth}{|l|X|l|}
        \hline
        \rowcolor{lightgray}
        \textbf{Composant} & \textbf{Technologie} & \textbf{Version} \\
        \hline
        Backend & Node.js & 18+ \\
        \hline
        Framework & Express.js & 4.18+ \\
        \hline
        Base de données & MongoDB & 6.0+ \\
        \hline
        ODM & Mongoose & 7.0+ \\
        \hline
        Authentification & JWT & 9.0+ \\
        \hline
        Validation & Joi & 17.0+ \\
        \hline
        Tests Unitaires & Jest & 29.0+ \\
        \hline
        Tests Property-Based & fast-check & 3.0+ \\
        \hline
        IA Locale & Ollama & 0.1+ \\
        \hline
        Modèle IA & Llama 3.2 & - \\
        \hline
        Email & Nodemailer & 6.9+ \\
        \hline
        PDF Parsing & pdf-parse & 1.1+ \\
        \hline
    \end{tabularx}
    \caption{Stack technologique}
    \label{tab:tech-stack}
\end{table}

\subsection{Configuration Requise}
\subsubsection{Exigences de Développement Local}
\begin{table}[H]
    \centering
    \begin{tabular}{|l|c|}
        \hline
        \rowcolor{lightgray}
        \textbf{Composant} & \textbf{Version requise} \\
        \hline
        Node.js & 18.x ou supérieur \\
        \hline
        npm & 9.x ou supérieur \\
        \hline
        MongoDB Community Edition & 6.0+ \\
        \hline
        Git & 2.40+ \\
        \hline
        Ollama & 0.1.20+ \\
        \hline
    \end{tabular}
    \caption{Exigences pour le développement local}
    \label{tab:local-requirements}
\end{table}

\subsubsection{Exigences Ollama}
\begin{table}[H]
    \centering
    \begin{tabular}{|l|c|}
        \hline
        \rowcolor{lightgray}
        \textbf{Paramètre} & \textbf{Valeur} \\
        \hline
        Mémoire minimale & 8 GB RAM \\
        \hline
        Stockage modèle & 4 GB \\
        \hline
        Version Ollama & 0.1.20+ \\
        \hline
        Port par défaut & 11434 \\
        \hline
        Timeout API & 30 secondes \\
        \hline
    \end{tabular}
    \caption{Exigences Ollama}
    \label{tab:ollama-requirements}
\end{table}

\section{API et Endpoints}
\label{sec:api}

\subsection{Authentification}
\begin{table}[H]
    \centering
    \begin{tabularx}{\textwidth}{|l|l|X|l|l|}
        \hline
        \rowcolor{lightgray}
        \textbf{Méthode} & \textbf{Endpoint} & \textbf{Description} & \textbf{Auth} & \textbf{Rôle} \\
        \hline
        POST & /api/auth/register & Inscription utilisateur & Non & - \\
        \hline
        POST & /api/auth/verify-otp & Vérification OTP & Non & - \\
        \hline
        POST & /api/auth/resend-otp & Renvoi OTP & Non & - \\
        \hline
        POST & /api/auth/login & Connexion & Non & - \\
        \hline
        GET & /api/auth/me & Profil utilisateur & Oui & Tous \\
        \hline
        POST & /api/auth/logout & Déconnexion & Oui & Tous \\
        \hline
        POST & /api/auth/refresh & Rafraîchir token & Oui & Tous \\
        \hline
    \end{tabularx}
    \caption{Endpoints d'authentification}
    \label{tab:auth-endpoints}
\end{table}

\subsection{Profils et Données}
\begin{table}[H]
    \centering
    \begin{tabularx}{\textwidth}{|l|l|X|l|l|}
        \hline
        \rowcolor{lightgray}
        \textbf{Méthode} & \textbf{Endpoint} & \textbf{Description} & \textbf{Auth} & \textbf{Rôle} \\
        \hline
        GET & /api/profile & Récupérer profil & Oui & Candidat \\
        \hline
        PUT & /api/profile & Mettre à jour profil & Oui & Candidat \\
        \hline
        POST & /api/profile/cv & Upload CV PDF & Oui & Candidat \\
        \hline
        GET & /api/companies & Lister entreprises & Oui & Tous \\
        \hline
        POST & /api/companies & Créer entreprise & Oui & Recruteur \\
        \hline
        PUT & /api/companies/:id & Mettre à jour entreprise & Oui & Recruteur \\
        \hline
        GET & /api/jobs & Rechercher offres & Oui & Tous \\
        \hline
        POST & /api/jobs & Créer offre & Oui & Recruteur \\
        \hline
        PUT & /api/jobs/:id & Mettre à jour offre & Oui & Recruteur \\
        \hline
        DELETE & /api/jobs/:id & Supprimer offre & Oui & Recruteur \\
        \hline
    \end{tabularx}
    \caption{Endpoints de données}
    \label{tab:data-endpoints}
\end{table}

\subsection{Fonctionnalités Avancées}
\begin{table}[H]
    \centering
    \begin{tabularx}{\textwidth}{|l|l|X|l|l|}
        \hline
        \rowcolor{lightgray}
        \textbf{Méthode} & \textbf{Endpoint} & \textbf{Description} & \textbf{Auth} & \textbf{Rôle} \\
        \hline
        GET & /api/wishlist & Récupérer wishlist & Oui & Candidat \\
        \hline
        POST & /api/wishlist/:jobId & Ajouter à wishlist & Oui & Candidat \\
        \hline
        DELETE & /api/wishlist/:jobId & Retirer de wishlist & Oui & Candidat \\
        \hline
        GET & /api/applications & Lister candidatures & Oui & Tous \\
        \hline
        POST & /api/applications & Soumettre candidature & Oui & Candidat \\
        \hline
        PUT & /api/applications/:id/status & Mettre à jour statut & Oui & Recruteur \\
        \hline
        POST & /api/ai/generate-cover-letter & Générer lettre & Oui & Candidat \\
        \hline
        POST & /api/ai/analyze-match & Analyser compatibilité & Oui & Recruteur \\
        \hline
        GET & /api/skills & Lister compétences & Oui & Tous \\
        \hline
        POST & /api/skills & Créer compétence & Oui & Admin \\
        \hline
        DELETE & /api/skills/:id & Supprimer compétence & Oui & Admin \\
        \hline
    \end{tabularx}
    \caption{Endpoints fonctionnalités avancées}
    \label{tab:advanced-endpoints}
\end{table}

\subsection{Administration}
\begin{table}[H]
    \centering
    \begin{tabularx}{\textwidth}{|l|l|X|l|l|}
        \hline
        \rowcolor{lightgray}
        \textbf{Méthode} & \textbf{Endpoint} & \textbf{Description} & \textbf{Auth} & \textbf{Rôle} \\
        \hline
        GET & /api/admin/users & Lister utilisateurs & Oui & Admin \\
        \hline
        GET & /api/admin/users/:id & Détails utilisateur & Oui & Admin \\
        \hline
        DELETE & /api/admin/users/:id & Supprimer utilisateur & Oui & Admin \\
        \hline
        GET & /api/admin/statistics & Statistiques plateforme & Oui & Admin \\
        \hline
        GET & /api/admin/logs & Logs système & Oui & Admin \\
        \hline
    \end{tabularx}
    \caption{Endpoints administration}
    \label{tab:admin-endpoints}
\end{table}

\section{Modèles de Données}
\label{sec:models}

\subsection{Schéma Utilisateur}
\begin{lstlisting}[language=JavaScript, caption=Modèle User, label=lst:user-model]
const userSchema = new mongoose.Schema({
  email: { 
    type: String, 
    unique: true, 
    required: true,
    lowercase: true,
    trim: true,
    match: [/^\S+@\S+\.\S+$/, 'Email invalide']
  },
  password: { 
    type: String, 
    required: true,
    minlength: 8 
  },
  role: { 
    type: String, 
    enum: ['candidate', 'recruiter', 'admin'], 
    required: true,
    default: 'candidate'
  },
  isVerified: { 
    type: Boolean, 
    default: false 
  },
  otp: {
    code: { 
      type: String 
    },
    expiresAt: { 
      type: Date 
    }
  },
  wishlist: [{ 
    type: mongoose.Schema.Types.ObjectId, 
    ref: 'JobOffer' 
  }],
  lastLogin: { 
    type: Date 
  },
  isActive: { 
    type: Boolean, 
    default: true 
  },
  createdAt: { 
    type: Date, 
    default: Date.now 
  },
  updatedAt: { 
    type: Date, 
    default: Date.now 
  }
}, {
  timestamps: true
});

// Index pour optimisation des requêtes
userSchema.index({ email: 1 });
userSchema.index({ role: 1, isActive: 1 });
userSchema.index({ createdAt: -1 });
\end{lstlisting}

\subsection{Schéma Candidature}
\begin{lstlisting}[language=JavaScript, caption=Modèle Application, label=lst:application-model]
const applicationSchema = new mongoose.Schema({
  job: { 
    type: mongoose.Schema.Types.ObjectId, 
    ref: 'JobOffer', 
    required: true,
    index: true
  },
  candidate: { 
    type: mongoose.Schema.Types.ObjectId, 
    ref: 'User', 
    required: true,
    index: true
  },
  status: { 
    type: String, 
    enum: ['pending', 'accepted', 'rejected', 'interview', 'withdrawn'], 
    default: 'pending',
    index: true
  },
  coverLetter: { 
    type: String,
    maxlength: 5000
  },
  aiGeneratedContent: { 
    type: String 
  },
  matchScore: { 
    type: Number,
    min: 0,
    max: 100
  },
  matchJustification: { 
    type: String,
    maxlength: 2000
  },
  notes: [{ // Notes internes du recruteur
    content: String,
    author: { type: mongoose.Schema.Types.ObjectId, ref: 'User' },
    createdAt: { type: Date, default: Date.now }
  }],
  timeline: [{ // Historique des changements de statut
    status: String,
    changedAt: { type: Date, default: Date.now },
    changedBy: { type: mongoose.Schema.Types.ObjectId, ref: 'User' }
  }],
  attachments: [{ // Fichiers joints supplémentaires
    filename: String,
    url: String,
    uploadedAt: { type: Date, default: Date.now }
  }],
  appliedAt: { 
    type: Date, 
    default: Date.now,
    index: true
  }
}, {
  timestamps: true
});

// Index composé unique pour éviter les doublons
applicationSchema.index({ job: 1, candidate: 1 }, { unique: true });

// Index pour les requêtes fréquentes
applicationSchema.index({ candidate: 1, appliedAt: -1 });
applicationSchema.index({ job: 1, status: 1, appliedAt: -1 });
applicationSchema.index({ matchScore: -1 });
\end{lstlisting}

\section{Sécurité}
\label{sec:securite}

\subsection{Stratégie d'Authentification}
\begin{figure}[H]
    \centering
    \begin{tikzpicture}[node distance=1.5cm, auto]
        % Start
        \node[startstop, fill=green!30] (start) {Utilisateur};
        
        % Register
        \node[process, fill=blue!30, below of=start] (register) {Inscription};
        
        % Send OTP
        \node[process, fill=orange!30, below of=register] (otp) {Envoi OTP};
        
        % Verify
        \node[process, fill=purple!30, below of=otp] (verify) {Vérification};
        
        % JWT
        \node[process, fill=red!30, below of=verify] (jwt) {Génération JWT};
        
        % API Access
        \node[io, fill=blue!30, below of=jwt] (api) {Accès API};
        
        % Arrows
        \draw[arrow] (start) -- (register);
        \draw[arrow] (register) -- (otp);
        \draw[arrow] (otp) -- (verify);
        \draw[arrow] (verify) -- (jwt);
        \draw[arrow] (jwt) -- (api);
        
        % Success feedback
        \draw[arrow, dashed] (api) -- ++(3,0) |- node[right] {Succès} (start);
    \end{tikzpicture}
    \caption{Flux d'authentification avec OTP}
    \label{fig:auth-flow}
\end{figure}

\subsection{Mesures de Sécurité}
\begin{table}[H]
    \centering
    \begin{tabularx}{\textwidth}{|l|X|}
        \hline
        \rowcolor{lightgray}
        \textbf{Mesure} & \textbf{Description} \\
        \hline
        Hachage de mots de passe & bcrypt avec salt rounds = 12 \\
        \hline
        Tokens JWT & Signés avec secret, expiration 24h \\
        \hline
        Hachage OTP & bcrypt avant stockage \\
        \hline
        Rate Limiting & 100 requêtes/minute par IP \\
        \hline
        CORS & Configuration stricte par origine \\
        \hline
        Helmet & Headers de sécurité HTTP \\
        \hline
        Validation d'entrée & Joi pour toutes les données \\
        \hline
        Sanitization & Protection contre les injections \\
        \hline
        Logging & Audit des actions sensibles \\
        \hline
    \end{tabularx}
    \caption{Mesures de sécurité implémentées}
    \label{tab:security-measures}
\end{table}

\section{Performance et Évolutivité}
\label{sec:performance}

\subsection{Métriques de Performance}
\begin{table}[H]
    \centering
    \begin{tabular}{|l|c|c|c|}
        \hline
        \rowcolor{lightgray}
        \textbf{Opération} & \textbf{Temps cible} & \textbf{95e percentile} & \textbf{Unité} \\
        \hline
        Requête API simple & < 100 & < 200 & ms \\
        \hline
        Génération lettre IA & < 30000 & < 45000 & ms \\
        \hline
        Analyse compatibilité & < 15000 & < 25000 & ms \\
        \hline
        Recherche offres & < 500 & < 1000 & ms \\
        \hline
        Upload CV & < 5000 & < 10000 & ms \\
        \hline
        Authentification & < 200 & < 500 & ms \\
        \hline
    \end{tabular}
    \caption{Métriques de performance}
    \label{tab:performance-metrics}
\end{table}

\subsection{Stratégies d'Optimisation}
\begin{itemize}
    \item \textbf{Caching:} Redis pour les données fréquemment accédées (optionnel en local)
    \item \textbf{Indexation:} Index MongoDB optimisés pour les requêtes courantes
    \item \textbf{Pagination:} Toutes les listes paginées avec limites
    \item \textbf{Compression:} Gzip pour les réponses API
    \item \textbf{Pool de connexions:} Optimisation des connexions MongoDB
    \item \textbf{Requêtes optimisées:} Projection Mongoose pour ne récupérer que les champs nécessaires
\end{itemize}

\section{Installation et Configuration Locale}
\label{sec:installation}

\subsection{Prérequis d'Installation}
\begin{enumerate}
    \item \textbf{Node.js et npm:}
    \begin{lstlisting}[language=bash]
# Installation sur Ubuntu/Debian
curl -fsSL https://deb.nodesource.com/setup_18.x | sudo -E bash -
sudo apt-get install -y nodejs

# Vérification
node --version
npm --version
    \end{lstlisting}
    
    \item \textbf{MongoDB:}
    \begin{lstlisting}[language=bash]
# Installation sur Ubuntu/Debian
wget -qO - https://www.mongodb.org/static/pgp/server-6.0.asc | sudo apt-key add -
echo "deb [ arch=amd64,arm64 ] https://repo.mongodb.org/apt/ubuntu focal/mongodb-org/6.0 multiverse" | sudo tee /etc/apt/sources.list.d/mongodb-org-6.0.list
sudo apt-get update
sudo apt-get install -y mongodb-org

# Démarrage du service
sudo systemctl start mongod
sudo systemctl enable mongod
    \end{lstlisting}
    
    \item \textbf{Ollama:}
    \begin{lstlisting}[language=bash]
# Installation Ollama
curl -fsSL https://ollama.ai/install.sh | sh

# Téléchargement du modèle Llama 3.2
ollama pull llama3.2

# Démarrage du service
ollama serve
    \end{lstlisting}
\end{enumerate}

\subsection{Configuration du Projet}
\begin{lstlisting}[language=bash, caption=Configuration locale du projet, label=lst:local-setup]
# 1. Cloner le dépôt
git clone https://github.com/votre-username/careerpath-ai.git
cd careerpath-ai/backend

# 2. Installer les dépendances
npm install

# 3. Configurer les variables d'environnement
cp .env.example .env
# Éditer le fichier .env avec vos configurations

# 4. Lancer la base de données MongoDB (si pas déjà démarré)
sudo systemctl start mongod

# 5. Lancer Ollama (si pas déjà démarré)
ollama serve

# 6. Démarrer le serveur de développement
npm run dev

# 7. Pour les tests
npm test              # Tests unitaires
npm run test:properties  # Tests property-based
npm run test:integration # Tests d'intégration
\end{lstlisting}

\subsection{Fichier de Configuration .env}
\begin{lstlisting}[caption=Exemple de fichier .env pour développement local, label=lst:env-example]
# Configuration CareerPath AI - Développement Local
NODE_ENV=development
APP_PORT=3000
APP_URL=http://localhost:3000

# MongoDB Local
MONGODB_URI=mongodb://localhost:27017/careerpath_development

# JWT
JWT_SECRET=votre_secret_jwt_pour_development
JWT_EXPIRES_IN=24h

# Ollama Local
OLLAMA_URL=http://localhost:11434
USE_MOCK_AI=false  # Mettre à true pour désactiver Ollama en développement

# Email (SMTP) - Configuration pour développement
SMTP_HOST=smtp.gmail.com
SMTP_PORT=587
SMTP_USER=votre_email@gmail.com
SMTP_PASS=votre_mot_de_passe_app

# Sécurité
RATE_LIMIT_WINDOW=15
RATE_LIMIT_MAX=100

# Upload
MAX_FILE_SIZE=10485760  # 10MB
UPLOAD_PATH=./uploads

# Logging
LOG_LEVEL=debug
LOG_TO_FILE=true
LOG_FILE_PATH=./logs/app.log
\end{lstlisting}

\section{Plan de Développement}
\label{sec:developpement}

\subsection{Environnements de Développement}
\begin{table}[H]
    \centering
    \begin{tabularx}{\textwidth}{|l|X|c|c|}
        \hline
        \rowcolor{lightgray}
        \textbf{Environnement} & \textbf{Description} & \textbf{Base de données} & \textbf{Monitoring} \\
        \hline
        Développement & Local avec Node.js natif & MongoDB local & Winston logs \\
        \hline
        Test & Tests automatisés & MongoDB Memory Server & Jest reports \\
        \hline
        Pré-production & Serveur de test local & MongoDB replica local & Basique \\
        \hline
    \end{tabularx}
    \caption{Environnements de développement local}
    \label{tab:dev-environments}
\end{table}

\subsection{Séquence de Développement}
\begin{enumerate}
    \item \textbf{Initialisation du projet}
    \begin{itemize}
        \item Installation des dépendances Node.js
        \item Configuration de MongoDB local
        \item Installation et configuration d'Ollama
        \item Mise en place de la structure de fichiers
    \end{itemize}
    
    \item \textbf{Développement des fonctionnalités}
    \begin{itemize}
        \item Implémentation des modèles Mongoose
        \item Développement des services métier
        \item Création des contrôleurs et routes
        \item Intégration de l'IA locale via Ollama
    \end{itemize}
    
    \item \textbf{Tests et validation}
    \begin{itemize}
        \item Tests unitaires avec Jest
        \item Tests property-based avec fast-check
        \item Tests d'intégration
        \item Validation manuelle des fonctionnalités
    \end{itemize}
    
    \item \textbf{Documentation et déploiement local}
    \begin{itemize}
        \item Documentation de l'API
        \item Guide d'installation local
        \item Configuration des variables d'environnement
        \item Tests de performance locale
    \end{itemize}
\end{enumerate}

\section{Maintenance et Support Local}
\label{sec:maintenance}

\subsection{Plan de Maintenance Locale}
\begin{table}[H]
    \centering
    \begin{tabularx}{\textwidth}{|l|X|c|}
        \hline
        \rowcolor{lightgray}
        \textbf{Activité} & \textbf{Description} & \textbf{Fréquence} \\
        \hline
        Backup base de données & Sauvegarde manuelle des données & Hebdomadaire \\
        \hline
        Mise à jour dépendances & Update packages npm & Mensuel \\
        \hline
        Nettoyage logs & Archivage et suppression anciens logs & Mensuel \\
        \hline
        Revue performance & Analyse et optimisation des performances & Trimestriel \\
        \hline
        Mise à jour Ollama & Update modèle et version Ollama & Semestriel \\
        \hline
        Audit sécurité & Revue de code et tests de sécurité & Annuel \\
        \hline
    \end{tabularx}
    \caption{Plan de maintenance pour développement local}
    \label{tab:local-maintenance}
\end{table}

\subsection{Dépannage Local}
\begin{itemize}
    \item \textbf{Problèmes MongoDB:}
    \begin{itemize}
        \item Vérifier que le service MongoDB est en cours d'exécution: \texttt{sudo systemctl status mongod}
        \item Redémarrer MongoDB: \texttt{sudo systemctl restart mongod}
        \item Vérifier les logs: \texttt{tail -f /var/log/mongodb/mongod.log}
    \end{itemize}
    
    \item \textbf{Problèmes Node.js:}
    \begin{itemize}
        \item Nettoyer le cache npm: \texttt{npm cache clean --force}
        \item Réinstaller les dépendances: \texttt{rm -rf node_modules package-lock.json \&\& npm install}
        \item Vérifier les permissions: \texttt{chmod -R 755 .}
    \end{itemize}
    
    \item \textbf{Problèmes Ollama:}
    \begin{itemize}
        \item Vérifier qu'Ollama est en cours d'exécution: \texttt{ollama list}
        \item Redémarrer Ollama: \texttt{pkill ollama \&\& ollama serve}
        \item Vérifier la connexion: \texttt{curl http://localhost:11434/api/version}
    \end{itemize}
\end{itemize}

\section{Roadmap Technique}
\label{sec:roadmap}

\begin{table}[H]
    \centering
    \begin{tabularx}{\textwidth}{|c|X|X|}
        \hline
        \rowcolor{lightgray}
        \textbf{Phase} & \textbf{Objectifs} & \textbf{Échéance} \\
        \hline
        \textbf{V1.0} & Fonctionnalités de base, IA locale & Q1 2024 \\
        \hline
        \textbf{V1.1} & Optimisations performance, caching local & Q2 2024 \\
        \hline
        \textbf{V1.2} & Notifications en temps réel (Socket.io) & Q3 2024 \\
        \hline
        \textbf{V1.3} & Analytics avancés pour recruteurs & Q4 2024 \\
        \hline
        \textbf{V2.0} & Interface administrateur enrichie & Q1 2025 \\
        \hline
        \textbf{V2.1} & Export/Import de données & Q2 2025 \\
        \hline
        \textbf{V2.2} & Machine Learning pour recommandations & Q3 2025 \\
        \hline
    \end{tabularx}
    \caption{Roadmap de développement local}
    \label{tab:roadmap}
\end{table}

\section{Conclusion}
\label{sec:conclusion}

\subsection{Récapitulatif}
CareerPath AI représente une solution complète et innovante pour le recrutement moderne. L'architecture choisie pour le développement local offre:
\begin{itemize}
    \item Une simplicité de déploiement avec des outils natifs
    \item Une sécurité renforcée avec authentification multi-facteurs
    \item Des performances optimisées pour un environnement local
    \item Une maintenance simplifiée sans dépendance à Docker
    \item Une évolutivité facilitée par une architecture modulaire
\end{itemize}

\subsection{Recommandations pour le Développement Local}
\begin{enumerate}
    \item Suivre scrupuleusement les étapes d'installation
    \item Configurer correctement les variables d'environnement
    \item Tester régulièrement toutes les fonctionnalités
    \item Maintenir à jour les dépendances et outils
    \item Sauvegarder régulièrement les données locales
\end{enumerate}

\subsection{Perspectives}
Le système est conçu pour évoluer vers:
\begin{itemize}
    \item L'intégration de modèles IA plus performants localement
    \item La prise en charge de plusieurs langues
    \item L'analyse prédictive des tendances du marché
    \item L'automatisation des processus de recrutement
    \item La possibilité de migration vers un environnement de production
\end{itemize}

\end{document}
